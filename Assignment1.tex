
\documentclass{report}
%%%%%%%%%%%%%%%%%%%%%%%%%%%%%%%%%%%%%%%%%%%%%%%%%%%%%%%%%%%%%%%%%%%%%%%%%%%%%%%%%%%%%%%%%%%%%%%%%%%%%%%%%%%%%%%%%%%%%%%%%%%%%%%%%%%%%%%%%%%%%%%%%%%%%%%%%%%%%%%%%%%%%%%%%%%%%%%%%%%%%%%%%%%%%%%%%%%%%%%%%%%%%%%%%%%%%%%%%%%%%%%%%%%%%%%%%%%%%%%%%%%%%%%%%%%%
\usepackage{amsfonts}
\usepackage{amsmath}

\setcounter{MaxMatrixCols}{10}
%TCIDATA{OutputFilter=LATEX.DLL}
%TCIDATA{Version=5.50.0.2960}
%TCIDATA{<META NAME="SaveForMode" CONTENT="1">}
%TCIDATA{BibliographyScheme=Manual}
%TCIDATA{Created=Thursday, February 15, 2018 00:51:26}
%TCIDATA{LastRevised=Thursday, February 15, 2018 02:05:11}
%TCIDATA{<META NAME="GraphicsSave" CONTENT="32">}
%TCIDATA{<META NAME="DocumentShell" CONTENT="Standard LaTeX\Standard LaTeX Report">}
%TCIDATA{CSTFile=40 LaTeX Report.cst}

\newtheorem{theorem}{Theorem}
\newtheorem{acknowledgement}[theorem]{Acknowledgement}
\newtheorem{algorithm}[theorem]{Algorithm}
\newtheorem{axiom}[theorem]{Axiom}
\newtheorem{case}[theorem]{Case}
\newtheorem{claim}[theorem]{Claim}
\newtheorem{conclusion}[theorem]{Conclusion}
\newtheorem{condition}[theorem]{Condition}
\newtheorem{conjecture}[theorem]{Conjecture}
\newtheorem{corollary}[theorem]{Corollary}
\newtheorem{criterion}[theorem]{Criterion}
\newtheorem{definition}[theorem]{Definition}
\newtheorem{example}[theorem]{Example}
\newtheorem{exercise}[theorem]{Exercise}
\newtheorem{lemma}[theorem]{Lemma}
\newtheorem{notation}[theorem]{Notation}
\newtheorem{problem}[theorem]{Problem}
\newtheorem{proposition}[theorem]{Proposition}
\newtheorem{remark}[theorem]{Remark}
\newtheorem{solution}[theorem]{Solution}
\newtheorem{summary}[theorem]{Summary}
\newenvironment{proof}[1][Proof]{\noindent\textbf{#1.} }{\ \rule{0.5em}{0.5em}}
\input{tcilatex}

\begin{document}

\title{Standard 
%TCIMACRO{\TeXButton{LaTeX}{\LaTeX{}} }%
%BeginExpansion
\LaTeX{}
%EndExpansion
Report}
\author{The Author}
\date{The Date}
\maketitle
\tableofcontents

\part{Topics In Finance: Assignment 1}

\bigskip 

\section{\protect\bigskip Introduction}

In this first assignment, it was asked to create two random samples X and Y
respectively, of 1000 observations from an i.i.d N(0,1) distribution where Y
has a 10\% correlation with X. Afterwards, a simple OLS regression of Y on X
should be provided and perform a regression analysis

\bigskip 

\section{Implementation}

By performing the following regression:

$Y=a+bX+e~$

\bigskip I get the following estimated equation:

$\overline{Y}~=-0.009+0.0996X$

$\ \ \ \ \ \ (0.0641)~\ (3.1742)$

\qquad $(0.4986)~\ (0.0015)$

$R^{2}=0.01$

$Root~MSE~=1.006$

\section{Conclusions}

\bigskip Taking into consideration the aforementioned estimates, the
following conclusions can be drawn:

1. Firstly, the coefficient of X is almost 0.01, as the correlation
coefficient between Y and X. This doesn't surprise us, as the coefficient $b$
in the univariate ols regression, is estimated as the correlation
coefficient, i.e. the ratio between the covariance of X and Y over the
variance of X. 

2. Secondly, this coefficient is statistical significant, with confidence
99.85\%

3. The constant term, $a$, is statistical insignificant, as the p-value of
the estimate is 0.5, i.e. I do not reject the null hypothesis that this term
is equal to 0.

4. $R^{2}~$is~0.01, which also makes sense, as we have a univariate
regression, meaning that Y is explained by X, by 10\%, which is the
correlation coefficient.

\end{document}
